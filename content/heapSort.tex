\section{Heap Sort}

\subsection{Introduction}
\textbf{History:} Heap Sort was first introduced by J.W.J. Williams in 1964. This innovation aimed to exploit the heap data structure to improve sorting performance.

\begin{center}
    \includegraphics[scale=.25]{img/jwjWilliams.jpg}\\
    J. W. J. Williams (1930-2012)
\end{center}

\textbf{Definition:} Heap Sort is a comparison-based, in-place sorting algorithm that leverages the properties of a heap data structure-specifically, a max heap when sorting in ascending order. Its basic premise is:
\begin{enumerate}
    \item \textbf{Max Heap Property:} In a max heap, every parent node is greater than or equal to its children. This guarantees that the largest element is always at the root.
    \item \textbf{In-Place Sorting:} Heap Sort rearranges the data within the input array, meaning it does not require significant additional memory.
    \item \textbf{Time Complexity:} It consistently operates in $O(n \log n)$ time in the worst case, making it reliable and predictable.
\end{enumerate}

\subsection{Algorithm}

\subsubsection{Main steps and Pseudocode}
\begin{itemize}
    \item Build the Max Heap
    \item Sort the Array (Extract Elements from Heap)
\end{itemize}

\begin{algorithm}
    \caption{Heapify}
    \begin{algorithmic}[1]
        \Function{Heapify}{$A$, $i$, $n$}
        \State $largest \gets i$
        \State $left \gets 2i + 1$
        \State $right \gets 2i + 2$
        \If{$left < n$ \textbf{and} $A[left] > A[largest]$}
        \State $largest \gets left$
        \EndIf
        \If{$right < n$ \textbf{and} $A[right] > A[largest]$}
        \State $largest \gets right$
        \EndIf
        \If{$largest \neq i$}
        \State \Call{Swap}{$A[i]$, $A[largest]$}
        \State \Call{Heapify}{$A$, $largest$, $n$}
        \EndIf
        \EndFunction
    \end{algorithmic}
\end{algorithm}

\begin{algorithm}
    \caption{BuildMaxHeap}
    \begin{algorithmic}[1]
        \Function{BuildMaxHeap}{$A$}
        \State $n \gets |A|$
        \For{$i \gets \lfloor n/2 \rfloor - 1$ \textbf{downto} 0}
        \State \Call{Heapify}{$A$, $i$, $n$}
        \EndFor
        \EndFunction
    \end{algorithmic}
\end{algorithm}

\begin{algorithm}
    \caption{HeapSort}
    \begin{algorithmic}[1]
        \Function{HeapSort}{$A$}
        \State \Call{BuildMaxHeap}{$A$}
        \For{$i \gets |A| - 1$ \textbf{downto} 1}
        \State \Call{Swap}{$A[0]$, $A[i]$}
        \State \Call{Heapify}{$A$, 0, $i$}
        \EndFor
        \EndFunction
    \end{algorithmic}
\end{algorithm}

\subsubsection{Detailed Implementation}

\textbf{Step 1: Build the Max Heap}
\begin{itemize}
    \item \textbf{Determine the Starting Point:}
    \begin{itemize}
        \item The array is treated as a complete binary tree.
        \item The last non-leaf node resides at index $\lfloor n/2 \rfloor - 1$, where $n$ is the total number of elements.
    \end{itemize}
    \item \textbf{Call heapify() on Each Non-Leaf Node:}
    \begin{itemize}
        \item Starting from the last non-leaf node and moving upward to the root, apply the heapify() procedure to enforce the max heap property.
        \item The heapify() procedure ensures that for a node at index $i$, the value at $i$ is greater than or equal to its children.
    \end{itemize}
\end{itemize}

\textbf{Step 2: Sort the Array (Extract Elements from Heap)}
\begin{itemize}
    \item \textbf{Swap the Root with the Last Element:}
    \begin{itemize}
        \item The root of the max heap (index 0) holds the maximum value. Swap it with the last element in the heap (at index $n-1$).
    \end{itemize}
    \item \textbf{Reduce Heap Size:}
    \begin{itemize}
        \item After the swap, consider the last element as sorted. Reduce the effective heap size by one so that it is excluded from further heap operations.
    \end{itemize}
    \item \textbf{Re-heapify the Root:}
    \begin{itemize}
        \item Call the heapify() procedure on the root (index 0) to restore the max heap property over the reduced heap.
    \end{itemize}
    \item \textbf{Repeat Extraction:}
    \begin{itemize}
        \item Continue this process—swapping the root with the last unsorted element and re-heapifying—until the entire array is sorted.
    \end{itemize}
\end{itemize}

\lstinputlisting[language=C++, caption=Heap Sort in C++, style=mystyle]{code/heapSort.cpp}

\subsection{Evaluation}

\textbf{Building the Heap}
\begin{itemize}
    \item \textbf{Operation:} Constructing a Heap (Max-Heap) from the input array.
    \item \textbf{Complexity:} $O(n)$
    \item \textbf{Explanation:} While a single “heapify” operation takes $O(\log n)$, the total time to build the heap is $O(n)$ due to optimization when handling elements at lower levels of the heap.
\end{itemize}

\textbf{Extracting Element from the Heap}
\begin{itemize}
    \item \textbf{Operation:} Extract the root element (Maximum) and rearrange the heap.
    \item \textbf{Complexity:} $O(n \log n)$
    \item \textbf{Explanation:} There are $n$ extractions, and each extraction is followed by a "heapify" operation that takes $O(\log n)$.
\end{itemize}

\textbf{In-Place Sorting}
\begin{itemize}
    \item \textbf{Operation:} Swap the extracted root element with the last unsorted element.
    \item \textbf{Complexity:} $O(n \log n)$
    \item \textbf{Explanation:} Swapping is combined with extraction and re-heapification, contributing to the overall complexity.
\end{itemize}

\textbf{Space Complexity}
\begin{itemize}
    \item \textbf{Space Used:} $O(1)$
    \item \textbf{Explanation:} Heap Sort is an in-place sorting algorithm, so it does not require additional space beyond the input array.
\end{itemize}

\subsection{Application}

Heap sort is a strong algorithm. It has several practical applications due to its efficiency and predictable performance:
\begin{enumerate}
    \item \textbf{Sorting Large Datasets:} Heap sort is particularly useful for sorting large datasets where memory usage is a concern. \textit{Example:} Sorting a large database of employee records by salary.
    \item \textbf{Priority Queue Implementation:} Heap sort is the foundation for implementing priority queues. Priority queues are used in scheduling algorithms, such as in operating systems or task management systems. \textit{Example:} Scheduling tasks in a CPU based on their priority levels.
    \item \textbf{External Sorting:} Heap sort is used in external sorting algorithms where data is too large to fit into memory. It helps in merging sorted chunks of data efficiently. \textit{Example:} Sorting large files stored on disk.
    \item \textbf{Real-Time Systems:} Heap sort is used in real-time systems where predictable performance is critical. \textit{Example:} Sorting sensor data in real-time for autonomous vehicles.
    \item \textbf{Selection Algorithms:} Heap sort is used in selection algorithms to find the $k$th smallest or largest element in a list. \textit{Example:} Finding the median of a dataset efficiently.
    \item \textbf{Game Development:} Heap sort is used in game development for managing and sorting game objects based on priority or distance. \textit{Example:} Rendering objects in a game based on their distance from the camera.
    \item \textbf{Event-Driven Simulations:} Heap sort is used in simulations where events need to be processed in a specific order. \textit{Example:} Simulating the behavior of a queue in a bank or a call center.
    \item \textbf{Operating Systems:} Heap sort is used in operating systems for memory management and process scheduling. \textit{Example:} Managing memory allocation for processes in a multi-tasking environment.
\end{enumerate}

\subsection{Problems}

\textbf{1. Kth Largest Element in an Array}
\begin{itemize}
    \item \textbf{Description:} Given an unsorted array, determine the $k$th largest element.
    \item \textbf{Detailed Instructions:}
    \begin{enumerate}
        \item Create a min-heap (priority queue) and insert the first $k$ elements of the array into it.
        \item For every remaining element in the array, compare it with the heap’s root. If it’s larger, remove the root and insert the new element, maintaining the heap property.
        \item Once all elements are processed, the root of the heap is the $k$th largest element. Use this technique to optimize your time complexity to $O(n \log k)$.
    \end{enumerate}
\end{itemize}

\textbf{2. Find Median from Data Stream}
\begin{itemize}
    \item \textbf{Description:} Continuously find the median of a stream of numbers.
    \item \textbf{Detailed Instructions:}
    \begin{enumerate}
        \item Use two heaps: a max-heap for the lower half of the numbers and a min-heap for the upper half.
        \item When a new number arrives, decide which heap to insert it into, then rebalance the heaps if their sizes differ by more than one.
        \item For median retrieval, if heaps are equal in size, average the roots; otherwise, return the root of the larger heap. This dual-heap method ensures that the median is found efficiently after each insertion.
    \end{enumerate}
\end{itemize}

\textbf{3. Merge k Sorted Lists}
\begin{itemize}
    \item \textbf{Description:} Merge multiple sorted linked lists into one sorted linked list.
    \item \textbf{Detailed Instructions:}
    \begin{enumerate}
        \item Initialize a min-heap and insert the head (first node) of each of the $k$ lists into it.
        \item Repeatedly extract the smallest node from the heap and append it to the merged list. If the extracted node has a next node, insert that node into the heap.
        \item Continue until the heap is empty. This process ensures that at each step, you’re adding the smallest current node, maintaining overall sorted order.
    \end{enumerate}
\end{itemize}

\textbf{4. Sliding Window Median}
\begin{itemize}
    \item \textbf{Description:} Given an array and a window size $k$, compute the median of every sliding window.
    \item \textbf{Detailed Instructions:}
    \begin{enumerate}
        \item Use two heaps (max-heap and min-heap) to represent the current window’s lower and upper halves.
        \item As the window slides, add the new element and remove the element that’s moving out of the window. Rebalance the heaps to maintain the required size properties.
        \item After every slide, compute the median based on the roots of the two heaps. This ensures an efficient update and retrieval of the median as the window moves.
    \end{enumerate}
\end{itemize}